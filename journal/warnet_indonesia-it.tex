\usemodule[pre-01]
\usemodule[tikz]
\usemodule[newmat]
\setupinteraction [state=start]
\starttext
\TitlePage { Warnet Indonesia (Italian)
\blank[3*medium]
\tfa Jaromil
 \blank[2*medium]
  \tfa Waag, October 2008, Amsterdam}\Topic {Internet in Indonesia}

L'Internet, arrivato  in Indonesia  all'inizio della crisi  degli anni
1990 e dopo trent'anni di  dittatura, si e' affermata economicamente e
politicamente come un mezzo al di fuori del controllo dello stato\footnote{Hill  D. and  Sen  K.   (2000) "The  Internet  in Indonesia's  New
Democracy"}

\placefigure[][]{}{\externalfigure[fotos/warnet_indonesia/indlogo.jpg]}

Quando il dittatore Suharto fu  costretto ad abdicare nel Maggio 1998,
alcuni giornalisti hanno tracciato un parallelo tra questo evento e la
presenza in rete del movimento Zapatista del Chiapas\footnote{Basuki  T. (1998) "Indonesia:  The Web  as a  Weapon", Development
Dialogue 2: The Southeast Asian Media in a Time of Crisis}, implicando che la
rivoluzione  indonesiana fosse facilitata  se non  addirittura guidata
dalla rete (Internet) \footnote{Marcus, D.  L. (1999).  "Indonesia  Revolt  Was  Net Driven"}

- \useURL[aa][http://www.progind.net][][http://www.progind.net] \from[aa]

- \useURL[aa][http://www.xs4all.nl/~peace][][http://www.xs4all.nl/\~peace] \from[aa]

\Topic {Gli Internet Cafe' (Warung / Warnet)}

\startquotation
Mentre  l'organizzazione  statale   dei  telefoni  pubblici  (Warung)
esisteva gia'  in altri  tempi e spazi,  il paradigma  degli Internet
Cafe (Warnet)  ha rapidamente gettato  le basi per la  sua espansione
quasi simultanea attraverso l'intero  tessuto sociale.  [\dots ]  I nodi
del cyber-spazio nello spazio reale  - Warnets ben simili ai Warung -
hanno  creato una rete  cosi' potente,  nel caso  dell'Indonesia alla
fine del  ventesimo secolo, da  essere piu' dinamica di  quella dello
stato, dell'economia e delle corporazioni. \footnote{Merlyna  Lim (2003)  The Internet,  Social Network  and  Reform in
Indonesia}
\stopquotation

\placefigure[][]{}{\externalfigure[fotos/warnet_indonesia/servis_elektronik.jpg]}

Il  sociologo  Castells  si  riferisce  a questo  fenomeno  come  alla
``preminenza  nel contesto sociale  della morfologia  sull'azione'', una
preminenza che e' la caratteristica principale delle societa' in rete.


\Topic {WSIS in Indonesia}

L'anno    2005   ha    portato   ad    uno    sviluppo   significativo
dell'infrastruttura telematica  in Indonesia: incentivi  allo sviluppo
uniforme sul territorio.

\placefigure[][]{}{\externalfigure[fotos/warnet_indonesia/child_with_computer.jpg]}

Il  Summit  Mondiale   sulla  Societa'  dell'Informazione  (WSIS)  nel
Dicembre  2003  ha  guidato  un'impatto  significativo  sul  piano  di
regolamentazione    statale    affinche'   l'Indonesia    raggiungesse
l'obiettivo  di  {\bf connettere  almeno  la  meta'  della  popolazione  ad
Internet entro il 2015}.


\Topic {Comunita' telematiche dal basso}

La  maggior  parte  delle  implementazioni  ad  oggi  sono  basate  su
comunita' auto-organizzate che  ricevono minimi aiuti esterni, incluso
il supporto del governo. Di seguito le piu' rilevanti\footnote{cfr.   Onno W.   Purbo, professore  all'Istituto di  Tecnologia di
Bangung}:

\placefigure[][]{}{\externalfigure[fotos/warnet_indonesia/people_reaching_tech.jpg]}

- \useURL[aa][http://www.groups.or.id][][http://www.groups.or.id] \from[aa]: forum liberi in  rete: nel 2004 da spazio a
piu' di 2005 gruppi di discussione con piu' di 65.000 partecipanti

-  \useURL[aa][http://www.ilmukomputer.com][][http://www.ilmukomputer.com] \from[aa]: pubblica manuali  e saperi  liberi che
riguardano l'informatica, sostenuto dai suoi stessi associati.

-  \useURL[aa][http://www.voipmerdeka.net][][http://www.voipmerdeka.net] \from[aa]:   la  ``Rebelnet''  che   da  servizi  di
telefonia in rete (VoIP) liberi  al popolo indonesiano: {\bf una delle piu'
larghe reti VoIP del mondo}

\Topic {Recenti cambiamenti}

{\bf Maggiori  liberta'}: il  5 gennaio  2005 un  atto del  ministero libera
l'uso delle frequenze wireless 2.4Ghz senza bisogno di alcuna licenza,
in  seguito a  svariati  raid delle  forze  dell'ordine ed  estenuanti
dibattiti pubblici ed al governo.

\placefigure[][]{}{\externalfigure[fotos/warnet_indonesia/win_piracy.jpg]}

{\bf Minori liberta'}: a meta'  del 2005 numerosi poliziotti irrompono negli
Internet  cafe' (Warnet)  di svariate  citta'  sequestrando computers,
arrestandone gli  operatori ed i proprietari  per un'operazione contro
la pirateria del software.


\Topic {Prospettive future}

\startquotation
Il  progressivo   avanzamento  delle  economie   corporative  suscita
preoccupazioni.  Questo  nuovo gigante accentratore  puo' depauperare
gli scambi  online dalle  originarie valenze politiche:  plasmando la
societa'  secondo la  logica del  consumo e  delle grandi  firme.  Di
conseguenza l'Internet diverra' un media neutrale ed omogeneo, la cui
funziona sara' unicamente appiattita  al commercio e alla pubblicita'
di beni consumistici. \footnote{Merlyna  Lim (2003) "The  Internet, Social  Network and  Reform in
Indonesia"}
\stopquotation

\placefigure[][]{}{\externalfigure[fotos/warnet_indonesia/corporate_gringos.jpg]}

\startquotation
Passando l'egemonia dallo  stato alle corporazioni l'Internet diviene
il passaggio  attraverso il quale  gli interessi privati  invadoni la
sfera pubblica. \footnote{Juergen  Habermas  (1991) "The  Structural  Transformation of  the
Public Sphere: An  Inquiry into a Category of  Bourgeois Society", MIT
Press.}
\stopquotation


\Topic {Reti attive}

Alcune organizzazioni umanitarie  acquisiscono un ruolo prominente nel
compilare   le  mappe   della  violenza   e   dell'ingiustizia:

- \useURL[bb][http://www.kontras.org/][][Kontras]\from[bb]\footnote{http://www.kontras.org/}: Spazio per solidarieta' e giustizia in favore di vittime

- \useURL[bb][http://www.ikohi.blogspot.com/][][IKOHI]\from[bb]\footnote{http://www.ikohi.blogspot.com/}: Commissione per gli scomparsi e le vittime di violenza

- \useURL[bb][http://www.wirantaprawira.de/ypkp/][][YPKP]\from[bb]\footnote{http://www.wirantaprawira.de/ypkp/}: Istituto Indonesiano per lo studio del massacro del 1965/1966

\placefigure[][]{}{\externalfigure[fotos/warnet_indonesia/ypkp.jpg]}

La diffusione di tecniche dei media locativi e tecnologie di mappatura
apre   la   possibilita'   di  completare   documentazioni   storiche,
approfondendo di conseguenza l'importanza di tali tecnologie.


\Topic {Terima Kasih}

\placefigure[][]{}{\externalfigure[images/dyne-big.png]}

Pagina web di Jaromil \useURL[aa][http://jaromil.dyne.org][][http://jaromil.dyne.org] \from[aa]

Diario indonesiano su \useURL[aa][http://jaromil.dyne.org/journal/indonesia.html][][http://jaromil.dyne.org/journal/indonesia.html] \from[aa]

Si ringrazia {\bf NLNET} per aver supportato questa ricerca.

Selamat jalan!



\stoptext
