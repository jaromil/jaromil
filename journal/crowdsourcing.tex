% Created 2010-07-19 Mon 12:05
\documentclass{beamer}
\usetheme{AnnArbor}
\usecolortheme{crane}
\useinnertheme[shadow]{rounded}
\usefonttheme{structuresmallcapsserif}
\institute[2010]{dyne.org / NIMk}
\usepackage[utf8]{inputenc}
\usepackage[T1]{fontenc}
\usepackage{fixltx2e}
\usepackage{graphicx}
\usepackage{longtable}
\usepackage{float}
\usepackage{wrapfig}
\usepackage{soul}
\usepackage{t1enc}
\usepackage{textcomp}
\usepackage{marvosym}
\usepackage{wasysym}
\usepackage{latexsym}
\usepackage{amssymb}
\usepackage{hyperref}
\tolerance=1000
\providecommand{\alert}[1]{\textbf{#1}}

\title{Crowdsourcing: il precariato digitale}
\author{Denis Roio (aka Jaromil)}
\date{Collegio di Milano, 27 Maggio 2010}

\begin{document}

\maketitle

\begin{frame}
\frametitle{Outline}
\setcounter{tocdepth}{3}
\tableofcontents
\end{frame}






\begin{frame}
\frametitle{Introduzione e scuse}
\label{sec-1}


Questa e' una pubblicazione prematura, seppur incoraggiata dal mio
supervisore Antonio Caronia, della mia ricerca di dottorato presso il

\textbf{Planetary Collegium (Universita' di Plymouth) M-Node}

La ricerca che sto svolgendo viene documentata in inglese, mi scuso
per la fretta, ma queste slides non sono ancora tradotte in italiano.
\end{frame}
\begin{frame}
\frametitle{Crowdsourcing}
\label{sec-2}


Il termine crowdsourcing\footnote{citazione da Wikipedia Italia } (da crowd, gente comune, e
outsourcing, esternalizzare una parte delle proprie attività\footnote{Marialuisa Pezzali, Crowdsourcing: quando la rete\ldots{} trova la
soluzione. «Il Sole 24 ORE», 2009. }) è
un neologismo che definisce un modello di business nel quale
un’azienda o un’istituzione richiede lo sviluppo di un progetto, di un
servizio o di un prodotto ad un insieme distribuito di persone non già
organizzate in un team.

Questo processo e' virtualizzato, avviene attraverso degli strumenti
web o comunque dei portali su internet.
\end{frame}
\section{Gratuitous and totally unaware crowdsourcing}
\label{sec-3}
\begin{frame}
\frametitle{Gratuitous crowdsourcing}
\label{sec-3_1}

\begin{itemize}
\item Gratis
\item Unaware
\item Working class
\end{itemize}
\end{frame}
\begin{frame}
\frametitle{Free ESP Games}
\label{sec-3_2}


Use the computational power of humans to perform a task that computers
cannot yet do by packaging the task as a ``game''.

  ``5000 people playing simultaneously an ESP game on image recognition
   can  label all images  on google  in 30  days. Individual  games in
   Yahoo!  and  MSN average over 5000  players at a  time.''\footnote{von Ahn, 2006 }

\includegraphics[width=10em]{images/esp_game.jpg}
\end{frame}
\begin{frame}
\frametitle{Massive Multiplayer Online RPG}
\label{sec-3_3}
\begin{itemize}

\item MMORPG\\
\label{sec-3_3_1}%
\includegraphics[width=10em]{images/wow_noob.jpg}

\begin{itemize}
\item World of Warcraft
\item Second Life
\item OpenSIM
\item etc.
\end{itemize}




\item Virtual reality architecture\\
\label{sec-3_3_2}%
\item Virtual miners\\
\label{sec-3_3_3}%
\end{itemize} % ends low level
\end{frame}
\begin{frame}
\frametitle{Electronic  Design  Automation  (EDA)}
\label{sec-3_4}



Complex problems are broke up into modules where the I/O of logic
circuits is tested against combinations computed by humans.

Problems can be solved by humans who are not even aware of how their
problem solving skills are then grouped and capitalised, while they
are assigned points that, once again, are worth nothing (see ESP
games).

\begin{itemize}
\item players unawareness
\item aleatory value
\item occult farming possible on SN (XSS and clustering on JS)
\end{itemize}
\end{frame}
\section{Mechanical Turks}
\label{sec-4}
\begin{frame}
\frametitle{Historical digression}
\label{sec-4_1}
\begin{itemize}

\item Mechanical Turks
\label{sec-4_1_1}%
\begin{itemize}
\item von Kempelen
\item Maelzel
\end{itemize}


\end{itemize} % ends low level
\end{frame}
\begin{frame}
\frametitle{von Kempelen's mechanical turk}
\label{sec-4_2}


  \includegraphics[width=10em]{images/turk1.jpg}

  On an autumn day in 1769, a Hungarian nobleman, Wolfgang von
  Kempelen, was summoned to witness a conjuring show at the imperial
  court of Maria Theresa, empress of Austria-Hungary. So unimpressed
  was Kempelen by what he saw that he impetuously declared that he
  could do better himself.  The following year Kempelen presented a
  mechanic man sitting behing a table: fashioned from wood, powered by
  clockwork, and dressed in a Turkish costume, it was capable of
  playing chess.
\end{frame}
\begin{frame}
\frametitle{Maelzel's chess player}
\label{sec-4_3}


  \includegraphics[width=10em]{images/mechanical_turk.jpg}

  ``Yet the question of its modus operandi is still
  undetermined. Nothing has been written on this topic which can be
  considered as decisive — and accordingly we find every where men of
  mechanical genius, of great general acuteness, and discriminative
  understanding, who make no scruple in pronouncing the Automaton a
  pure machine, unconnected with human agency in its movements, and
  consequently, beyond all comparison, the most astonishing of the
  inventions of mankind.''\footnote{Poe, E. A. (1836) Maelzel's Chess-Player }
\end{frame}
\section{Cheap, partially aware crowdsourcing}
\label{sec-5}
\begin{frame}
\frametitle{Mechanical turks}
\label{sec-5_1}


 \includegraphics[width=10em]{images/mturk.jpg}

\begin{itemize}
\item artificial artificial life
\item repetitive tasks that will be automated in future
\item humans are behind pseudo-automation
\item serialised silly  tasks
\end{itemize}
\end{frame}
\begin{frame}
\frametitle{Mompreneur}
\label{sec-5_2}


  \includegraphics[width=10em]{images/liveOps.jpg}

\begin{itemize}
\item Employees are contractors passing uniform tests
\item Menu to choose where to go and act with your headset
\item Menu option starts expanding as their success is proven
\end{itemize}

 
\begin{itemize}

\item Problems\\
\label{sec-5_2_1}%
\begin{itemize}
\item Problematic applications are hidden
\item Increasing false positive cases
\item Fatal errors
\end{itemize}


\end{itemize} % ends low level
\end{frame}
\begin{frame}
\frametitle{"Rehabilitation" / Brain scooping}
\label{sec-5_3}


 \includegraphics[width=10em]{images/samasource.jpg}

\begin{itemize}
\item \href{http://www.samasource.org}{Samasource} provides work  for people  in ``refugee camps''
\item Work that can be done on existing infrastructure
\item Look in a screen and push buttons for 1 penny
\end{itemize}
\begin{itemize}

\item Problems
\label{sec-5_3_1}%
\begin{itemize}
\item peripheral fordism exploitation
\item human alienation
\end{itemize}


\end{itemize} % ends low level
\end{frame}
\begin{frame}
\frametitle{Art with MTURKS}
\label{sec-5_4}


\begin{itemize}
\item The Sheep Market

\begin{itemize}
\item 10.000 sheeps created by online workers
\item \href{http://www.thesheepmarket.com/}{http://www.thesheepmarket.com/}
\end{itemize}

\end{itemize}
\end{frame}
\begin{frame}
\frametitle{The faces of Turks}
\label{sec-5_5}


\includegraphics[width=10em]{images/turk_faces.jpg}
\end{frame}
\section{Expensive, providence based crowdsourcing}
\label{sec-6}
\begin{frame}
\frametitle{Prize based}
\label{sec-6_1}


\begin{itemize}
\item Expensive
\item Top down
\item Centralized
\end{itemize}
\end{frame}
\begin{frame}
\frametitle{Internal Prize}
\label{sec-6_2}


  \includegraphics[width=10em]{images/innocentive.png}

\begin{itemize}
\item Best wins selection
\item Project to be produced by the company giving the prize
\item No risk for winning participants
\item smaller prizes
\end{itemize}
\begin{itemize}

\item Problems
\label{sec-6_2_1}%
\begin{itemize}
\item Deregulated appropriation
\item Enclosure
\end{itemize}



\end{itemize} % ends low level
\end{frame}
\begin{frame}
\frametitle{External Prize}
\label{sec-6_3}


  \includegraphics[width=10em]{images/XPrize.png}

\begin{itemize}
\item Declared as philantropic
\item Investement to win results cost only  partially
\item Future revenue comes from independent distribution of product
\end{itemize}
\begin{itemize}

\item Problems
\label{sec-6_3_1}%
\begin{itemize}
\item High cost failure
\end{itemize}



\end{itemize} % ends low level
\end{frame}
\section{Difference from Free and Open Source Software}
\label{sec-7}
\begin{frame}
\frametitle{This is not Free and Open Source Software}
\label{sec-7_1}


The dynamics of Free and Open Source Software\footnote{as defined by the \href{http://gnu.org}{GNU Project} and the \href{http://fsf.org}{Free Software Foundation} } are
fundamentally different, since they apply logics of collaboration to
\textbf{sharable goals and products} that can be adapted and redistributed
infinitely in different contexts, in a \textbf{decentralized fashion}.

\begin{itemize}
\item Free cooperation vs. proprietary competition
\end{itemize}


The coalitions of those humans that are investing labour and the
groups connected to them are aimed primarily at the quality of their
\textbf{cooperative efforts} and the peer review of those who are adopting
them.

The redistribution of the (digitally reproducible) artifacts is left
uncontrolled, creations can spawn recursively and muliply their
virtue.
\end{frame}
\section{Conclusion}
\label{sec-8}
\begin{frame}
\frametitle{References}
\label{sec-8_1}


There isn't much literature yet around the issue of labour politics in
the digital environment, an exploration is offered by this book:

\begin{itemize}
\item Zittrain, J. (2009) The Future of the Internet, Yale Press
\end{itemize}


The \href{http://planet.dyne.org}{dyne.org network} and the research publication \href{http://www.thenextlayer.org/}{The Next Layer} offer
good insights to understand the condition of \textbf{digital natives} in
these and more scenarios.
\end{frame}
\begin{frame}
\frametitle{Future scenarios}
\label{sec-8_2}


\begin{itemize}
\item Global dimension of labour, i.e. \href{http://userlabor.org}{User Labor} markup (ULML)
\item Exasperated import/export labour regulations
\item Human exploitation in virtual environments
\item Financial factors hit global labour markets
\end{itemize}
\end{frame}
\begin{frame}
\frametitle{Salaam/Shalom/Shanthi/Dorood/Peace}
\label{sec-8_3}


Jaromil's musings on \href{http://jaromil.dyne.org/journal}{http://jaromil.dyne.org}

\begin{itemize}
\item email: jaromil@dyne.org
\item IM: xmpp:jaromil@hinezumi.im
\item \href{http://freaknet.org}{Freaknet} Medialab / \href{http://dyne.org}{dyne.org} network
\item \href{http://nimk.nl}{Netherlands Media Art Institute} / R \& D
\item \href{http://www.planetary-collegium.net/}{Planetary Collegium} / \href{http://www.planetary-collegium.net/}{M-Node} / PhD candidate
\end{itemize}


Brainstorming thanks to \#core, z4nn4, tomak, crash, merijn, /dev/mat,
nightolo, acme, tati, acracia, asbesto `n ze otehrs

\includegraphics[width=10em]{images/dyne-big.png}
\end{frame}

\end{document}
