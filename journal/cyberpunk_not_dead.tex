\usemodule[pre-01]
\usemodule[tikz]
\usemodule[newmat]
\setupinteraction [state=start]
\starttext
\TitlePage { Cyberpunk is not dead
\blank[3*medium]
\tfa Jaromil
 \blank[2*medium]
  \tfa GOGBOT, September 2008, Enschede   Plumberconf, August 2009, Vienna}\Topic {Cyberpunk}

\placefigure[][]{}{\externalfigure[images/clockwork_skull_sm.jpg]}

- {\bf Hybrid phenomenon}: scientific and humanistic aspects merge

- {\bf Criticism to Technocracy}, grassroots movements reclaiming autonomy

- {\bf Immaterial values} are well understood, as identity or participation

- {\bf Information is Power} and it wants to be free, by its own nature

- {\bf Industrialisation is a virus} tainting the future of humankind

Video excerpt \useURL[bb][http://jaromil.dyne.org/journal/videos/lain-thewired-excerpt.ogm][][Lain - Serial Experiments - episode 05]\from[bb]\footnote{http://jaromil.dyne.org/journal/videos/lain-thewired-excerpt.ogm}


\Topic {Chronology}

- {\bf The Prototype Era (Pre 1980)} words like computer, robot, cyborg, and
punk  enter   dictionaries;  Late  60--70s   counterculture;  Kraftwerk
electronic music.

- {\bf The Golden Era (1980--1993)}  Neuromancer is published; the Golden age
of video  games on Commodore, Atari  and later IBM  PCs; Blade Runner;
NYC  black-out due  to  programming error;  Misguided Secret  Services
(Sundevil, Gladio); The WWW goes public; the GNU project is founded.

- {\bf The Mainstream Era  (1993--1999)} Time Magazine; Lawnmower Man, Johnny
Mnemonic and  more movies; Web population  explodes; Microsoft Windows
becomes dominant OS.

- {\bf The New  Millennium Era (Now)} The Matrix;  George Bush; 9/11 Patriot
Act;  DRM  developed  and  hacked;  Google; Web  2.0  buzzword;  Spam,
botnets, more  viruses; nanotechnology and  cybernetic implants; RFID,
global  surveillance;   data-mining;  alternative  grassroots   OS  as
GNU/Linux and BSD grow in importance.

(more on \useURL[bb][http://www.cyberpunkreview.com/movie/essays/the-four-eras-of-cyberpunk/][][cyberpunk review]\from[bb]\footnote{http://www.cyberpunkreview.com/movie/essays/the-four-eras-of-cyberpunk/})



\Topic {Role playing game}

\placefigure[][]{}{\externalfigure[images/cyberpunk2020_big.jpg]}

On the tracks of Dungeon \& Dragons success, the passion for RPG starts
affecting the imaginary of young generations in 1980 until today.



\Topic {Time magazine in 1993}

\placefigure[][]{}{\externalfigure[images/cyberpunk_on_time_1993.jpg]}

The popularity tames down  socio-political aspects in public, a common
pattern for underground cultures emerging as popular.


\Topic {Culture jammers}

Filmmaker Craig Baldwin says in \useURL[bb][http://vader.inow.com/~sam/cultjam2.html][][Culture Jamming 2.0]\from[bb]\footnote{http://vader.inow.com/~sam/cultjam2.html}:

\blank[medium]\hrule\blank[medium]

\startquotation
``Jujitsu  is  the art  of  using the  weight  of  the enemy  against
itself,''  Baldwin explains. ``With  corporations, sometimes  the only
way to beat them is not by brute force, but by {\bf symbolic agility}.''
\stopquotation

\blank[medium]\hrule\blank[medium]

\placefigure[][]{}{\externalfigure[images/systemshock_sm.jpg]}

Symbolic  Agility  is  a   property  of  independent,  autonomous  and
non-systemic agents  that, in  a society of  total control,  can still
{\bf operate as humans} and {\bf inspire other humans}.

Video excerpt \useURL[bb][http://jaromil.dyne.org/journal/videos/brazil-excerpt.ogm][][Brazil - Mr.Tuttle clandestine help]\from[bb]\footnote{http://jaromil.dyne.org/journal/videos/brazil-excerpt.ogm}


\Topic {Self empowered rebels}

Social injustice and systemic failure  are crucial in the dialectic of
cyberpunk, as in other  popular movements: graffiti, skaters, hip-hop,
reggae, classic punk.

As new generations empathise better with technologies, their rebellion
becomes disproportional  and can  affect the systems  functionality in
unexpected ways.

\placefigure[][]{}{\externalfigure[images/embrio_cyberpunk_sm.jpg]}

Video excerpt \useURL[bb][http://jaromil.dyne.org/journal/videos/lain-knightsuicide-excerpt.ogm][][Lain - Serial Experiments - episode 10]\from[bb]\footnote{http://jaromil.dyne.org/journal/videos/lain-knightsuicide-excerpt.ogm}


\Topic {Black hat / White hat}

- {\bf Pars Destruens}: viruses, reverse engineering, information leaks

- {\bf Pars Construens}: free software, independent media, sharing

\placefigure[][]{}{\externalfigure[images/BadTripBob_sm.jpg]}



\Topic {Internet natives}

\startquotation
Open-source software was nominated as worthy of protection by the UN
during the World Summit on the Information Society in 2003.
\stopquotation

\placefigure[][]{}{\externalfigure[images/BadTripWalls_sm.jpg]}

There is an  analogy between online and onsite  natives, in both cases
colonised by mega-corporations and industrial speculators.



\Topic {Planetary hybrid future}

As the  digital divide shrinks, cyberpunk  spreads rapidly hybridising
with different cultures in Asia, Africa and South America.

\placefigure[][]{}{\externalfigure[images/nirvana_sm.jpg]}


\Topic {Rastasoft}

\placefigure[][]{}{\externalfigure[images/rastalion.jpg]}

{\bf Proprietary  software spreads the  dependence from  business companies
thru the  populace}: whenever we  share our knowledge  on how to  use a
certain software,  we make the  people in need  to buy the  tools from
merchants in order to express their Creativity.

{\bf The  roots of Rasta  culture can  be found  in Resistance  to slavery}.
This software is  not a business.  This software is  {\bf free as in speech}
and is one step in the struggle for Redemption and Freedom.

\placefigure[][]{}{\externalfigure[images/rastasoft.jpg]}


\starttyping
while ( love & passion ) {
  for( fight = 0 ; rights < freedom ; rights++ )
    fight = standup( rights );
  free( babylon );
}
\stoptyping


\Topic {Thanks!}

\placefigure[][]{}{\externalfigure[images/dyne-big.png]}

Jaromil's musings on \useURL[aa][http://jaromil.dyne.org][][http://jaromil.dyne.org] \from[aa]

A thousand flowers will blossom!



\stoptext
