\usemodule[pre-01]
\usemodule[tikz]
\usemodule[newmat]
\setupinteraction [state=start]
\starttext
\TitlePage { FreeJ Vision Mixer
\blank[3*medium]
\tfa Jaromil
 \blank[2*medium]
  \tfa Digitale Pioners}\Topic {Brewed in C++ since 2001}

\placefigure[][]{FreeJ logo by Sciatto Prod.}{\externalfigure[images/ipernav.png]}

Following the motto:  {\bf set the veejay free!} FreeJ is  the first GNU GPL
``veejay'' software, also aiming at  a 100\% free video plugin framework,
together with Fukuchi Kentaro's EffecTV.


\Topic {Realtime video mixer and linear video editor}

FreeJ is a digital instrument for realtime video manipulation used in
the fields of dance theater, veejaying, medical visualization and TV.

FreeJ  lets  you interact  with  multiple  layers  of video  (images,
movies,   live  cameras  and   streams,  particle   generators,  text
scrollers, flash animations and more), filter them with effect chains
and  then mix  them together.

It  can be  operated live  from  a text  console and  scripted to  be
controlled by keyboard, midi, OSC and joysticks.


\Topic {Core features}

- Interoperable: code can be reused and plugged in applications

- Efficient: faster than usual multimedia frameworks for desktops

- Computes in parallel: designed for multi-threading since the beginning

- Modular: makes use of existing libraries, plugs to different software

- It's 100\% FREE: GNU GPL v3, listed in the FSF software directory


\Topic {Developers and contributions}

\placefigure[][]{The FreeJ team at Piksel 2005 in Norway, from left to right: Kysucix, Mr.Goil and Jaromil}{\externalfigure[fotos/portraits/freej_piksel2005.jpg]}


Code included:

- Sam Lantinga (sdl\textunderscore{}*)

- FFMpeg video decoders

- Andreas Schiffler (sdl\textunderscore{}gfx)

- Jan (theorautils)

- Dave Griffiths (audio bus)

- Nemosoft (ccvt)

- Charles Yates (yuv2rgb)

- Steve Harris (liblo)

- Olivier Debon (flash).


\Topic {Backend engine}

\placefigure[][]{}{\externalfigure[images/freej_serverside.png]}


\Topic {Easy to Deploy}

Binary  packages   are  maintained   for  the  main   repositories  of
Debian/Ubuntu, Fedora and more GNU/Linux distributions.

Debian package subdivision:

- freej

- freej-dbg

- freej-doc

- libfreej

- libfreej-dev

- python-freej

- python-freej-dbg



\Topic {Tagtool points of interest}

- 100\% Free - GNU GPL version 3

- Cross Platform (32/64 bit, big/little endian, ARM support)

- Extensible Layer and Controller API

- Scriptability in Python and Javascript

- OSX support (Native Cocoa/Carbon and QT components)

- Encoding and Streaming support (Ogg/Theora)

- Controllers: Wiimote, Mouse \& Joy, OSC, MIDI, serial and text Console

- Highly efficient and documented code with reliable dependencies



\stoptext
