\usemodule[pre-04]
\usemodule[tikz]
\usemodule[newmat]
\setupinteraction [state=start]
\starttext
\TitlePage { Internet of Things
\blank[3*medium]
\tfa Jaromil
 \blank[2*medium]
  \tfa Institute of Network Cultures, October 2008, Amsterdam}\Topic {``Free'' market lies}

\startitemize
\item When you purchase an object you own it, no strings attached
\item Competition is the base for evolution, enhancing quality
\item Monopolies are the enemies of market efficiency
\stopitemize

\placefigure[][]{}{\externalfigure[images/brokers1.jpg]}


\Topic {Historical failures}

\startitemize
\item Digital copy-protection
\item Fear based societies
\item Equal opportunities
\item Democracy and Education
\stopitemize

\placefigure[][]{}{\externalfigure[images/brokers2.jpg]}


\Topic {Hypocritical attempts}

\startitemize
\item Intellectual property
\item Creative Industries
\item Corporate service providers
\item Semi-public research funds
\item Corporate responsibility
\item Philanthropic bubbles
\stopitemize

\placefigure[][]{}{\externalfigure[images/brokers3.jpg]}


\Topic {Viable alternatives}

\startitemize
\item Empower content producers
\item Peer to peer service provision
\item Local ownership of production means
\item More how-to, less manifesto
\stopitemize

\placefigure[][]{}{\externalfigure[images/bricolabs-sm.jpg]}

\startquotation
A  global platform  to investigate  the  new {\bf loop  of open  content,
software  and hardware for  community applications},  bringing people
together with new  technologies and distributed connectivity, unlike
the  dominant focus  of IT  industry on  security,  surveillance and
monopoly of information and infrastructures.
\stopquotation


\Topic {Success stories}

\startitemize
\item TV repair shops
\item Free VOIP telephony
\item Micro breweries
\item Free Software (where {\bf free is for real})
\stopitemize

\placefigure[][]{}{\externalfigure[images/weaverbirdsnest.jpg]}


\Topic {Design for Commoning}

It is important to recognise that the  end goal of our work is not the
creation  of  novel  devices,   new  markets,  or  emergent  forms  of
communicative activities.   Instead, our goal is the  development of a
{\bf generic  information  infrastructure} -  and  the  tools and  knowledge
required to use, appropriate, rework,  and innovate it.

We  see such  a resource,  created, maintained,  and owned  by  no one
individual, institution, or company, as  a necessary part of a {\bf digital
territory} in which a  global, informed, and socially engaged citizenry
has the  potential to  develop and  grow. Such a  terrain is  not just
about  the technology;  equally important  is the  development  of the
{\bf critical making} skills that allow  this citizenry to engage with their
information landscape, to make educated political and economic as well
as technical decisions.

The  coming  decade  worldwide  will  be determined  by  the  {\em strained
relationship between formal  and informal structures and environments}.
A {\bf design for  commoning} one that views {\bf uncommon  ground as a resource},
rather than a  threat is the way towards living  together locally in a
globally connected world.

( Matt Ratto )



\Topic {Salaam/Shalom/Shanthi/Dorood/Peace}

\placefigure[][]{}{\externalfigure[images/dyne-big.png]}

More musings on \useURL[aa][http://jaromil.dyne.org/journal][][http://jaromil.dyne.org/journal] \from[aa]

R\&D at the Netherlands Media Art Institute (NIMK)

Thanks, a thousand flowers will blossom!



\stoptext
