\usemodule[pre-01]
\usemodule[tikz]
\usemodule[newmat]
\setupinteraction [state=start]
\starttext
\TitlePage { Coltan and Blood
\blank[3*medium]
\tfa Jaromil
 \blank[2*medium]
  \tfa Transmediale, January 2009, Berlin}\Topic {The Coltan}

Coltan is an abbreviation for {\bf Columbite-Tantalite}.  A mineral with the
unique  ability when  refined to  reduce corrosion  and  increase heat
resistance.

\placefigure[][]{}{\externalfigure[images/coltan_goma_1.jpg]}

The mineral is valuable for technology industries, its demand exploded
in the past 10 years.  It is used to produce {\bf cell phones, dvd-players,
laptop  computers  and games}\footnote{the Play Station 2 production  was halted in December 2000 because
of coltan shortage}  as  well  in  {\bf jet engines,  ballistic
missiles and nuclear reactors}.

{\bf 80\% World's known  reserves are in Congo} (DRC),  mostly in the Eastern
provinces of {\bf Kivus and Orientale},  extracted from the ore in a process
similar to that of the Californian gold miners of the 1800s.

\Topic {The Blood}

The  war in Eastern  Congo is  mainly driven  by economic  factors: an
outside consumer  market hungry for  {\bf laptops and telephones}  for which
Congo delivers the raw materials.

\placefigure[][]{}{\externalfigure[images/coltan_goma_2.jpg]}

The conflict  has resulted  in estimated {\bf 4  million deaths},  {\bf 2 million
displaced} and  {\bf 340.000 refugees}  in neighbouring countries.

{\bf 30\% of  East Congolese school children}  are reported to  be working in
the coltan mines under inhuman conditions.

This crisis is the worst contemporary humanitarian crisis in the World
provoked by Western post-colonialist economies.


\Topic {The Invasion}

In 1998 leaders of Rwanda and Uganda, military trained and equipped by
foreign countries, invaded the mineral-rich areas of Congo (DRC).

\placefigure[][]{}{\externalfigure[images/coltan_militars_1.jpg]}

The  invaders installed colonial-style  governments supported  in arms
and training  by Western allies,  together with a \$5  million Citibank
loan. Invasion of DRC  allowed corporations as {\bf American Mineral Fields}
(AMF) to illegally mine.  San Francisco based engineering firm {\bf Bechtel
Inc.} established  ties, drawing up  an inventory of  mineral resources
and providing the industry with refined supplies.\footnote{Project Censored, 2003 Top 25 stories, "American Companies Exploit
the Congo", Philip Beard, Arinze Anoruo, Chris Salvano.}

\Topic {The Business}

As of today  the mineral exploitation is known  to involve U.S.  based
{\bf Cabot Corp.  and OM Group}; Germany's {\bf HC Starck} and Chinese corporation
{\bf Nigncxia}.\footnote{Dollars and Sense,  July/August 2001, "The Business of  War in the
Democratic Republic Of Congo:  Who benefits?", by Dena Montague and
Frieda Berrigan}

\placefigure[][]{}{\externalfigure[images/coltan_militars_2.jpg]}

Payback  to Congolese  militias is  mostly  in arms,  from dealers  as
{\bf Simax}, {\bf Lockheed Martin}, {\bf Halliburton},  {\bf Northrop Grumman}, {\bf GE}, {\bf Boeing} and
{\bf Raytheon}.   Their action is  covert under  the face  of ``humanitarian''
organizations as  {\bf CARE} (funded  by Lockheed Martin)  and {\bf International
Rescue Committee}.\footnote{CovertAction Quarterly, Summer 2000, "U. S. Military and Corporate
Recolonization of the Congo", by Ellen Ray}

\Topic {Our Plan}

Most international  media still portrays the  war in East  Congo as an
ethnic conflict,  marginalizing efforts  made by local  communities to
find a solution.

\placefigure[][]{}{\externalfigure[images/yole_logo.png]}

The indigenous  Congolese organization  {\bf Yole!Africa calls  everyone to
stop using laptops,  phones or I-pods on day July 2}  to tell the World
that many  people, most unknowingly, are responsible  for this ongoing
catastrophe every time they buy a new laptop, phone or I-pod.

\blank[medium]\hrule\blank[medium]

The  hacker's  network {\bf dyne.org  calls  high-tech  industries to  free
products from DRM, Trusted Computing and any other restriction} so that
everyone  in  the  World is  able  to  appropriate  the use  of  these
technologies and build local businesses.

\blank[medium]\hrule\blank[medium]

Yole!Afrika  is co-producing  a  documentary about  mobile phones  and
blood directed by Frank Polsen, out this year.


\Topic {Salaam/Shalom/Shanthi/Dorood/Peace}

This  documentation  was  made  possible  thanks  to  the  efforts  of
\goto{Yole!Africa}[program(www.alkebu.org)]\footnote{www.alkebu.org}  (Democratic   Republic  of  Congo),   \useURL[bb][http://www.baobabconnections.org][][Baobab  Connections]\from[bb]\footnote{http://www.baobabconnections.org}
(Amsterdam),  \useURL[bb][http://www.projectcensored.com][][Project  Censored]\from[bb]\footnote{http://www.projectcensored.com}  (USA),  the \useURL[bb][http://dyne.org][][dyne.org  hackers]\from[bb]\footnote{http://dyne.org}  and  a
spontaneous human alliance against organized crime, military abuse and
corporate exploitation.

\blank[medium]\hrule\blank[medium]

Pictures taken by photographer \useURL[bb][http://www.riccardogangale.com][][Riccardo Gangale]\from[bb]\footnote{http://www.riccardogangale.com}

\blank[medium]\hrule\blank[medium]

Thanks  go to  Petna Ndaliko  Katondolo,  Yehudi van  de Pol,  Patrice
Riemens, Gnumark and Ecila for their support and precious suggestions.

\blank[medium]\hrule\blank[medium]

The  ``Coltan  and Blood''  slides  are  available  on the  Internet  on
\useURL[bb][http://jaromil.dyne.org/journal][][Jaromil's Journal]\from[bb]\footnote{http://jaromil.dyne.org/journal} website, feel free to spread and re-use!

\blank[medium]\hrule\blank[medium]

Thanks, a thousand flowers will blossom!



\stoptext
