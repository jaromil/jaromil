\usemodule[pre-01]
\usemodule[tikz]
\usemodule[newmat]
\setupinteraction [state=start]
\starttext
\TitlePage { Warnet Indonesia (English)
\blank[3*medium]
\tfa Jaromil
 \blank[2*medium]
  \tfa Waag, October 2008, Amsterdam}\Topic {Internet in Indonesia}

The Internet,  which came to Indonesia  during the early  phase of the
political  crisis  in  the  1990s,  has risen  both  economically  and
politically to become an alternative medium that has found its way out
from under the control of the state\footnote{Hill  D. and  Sen  K.   (2000) "The  Internet  in Indonesia's  New
Democracy"}

\placefigure[][]{}{\externalfigure[fotos/warnet_indonesia/indlogo.jpg]}

When Suharto,  was forced  to step  down in May  1998, writers  drew a
parallel  between  this  event  and  the  Zapatista  Net  movement  in
Chiapas\footnote{Basuki  T. (1998) "Indonesia:  The Web  as a  Weapon", Development
Dialogue 2: The Southeast Asian Media in a Time of Crisis}, implying  that the  political revolution in  Indonesia was
(Inter)Net-driven \footnote{Marcus, D.  L. (1999).  "Indonesia  Revolt  Was  Net Driven"}

- \useURL[aa][http://www.progind.net][][http://www.progind.net] \from[aa]

- \useURL[aa][http://www.xs4all.nl/~peace][][http://www.xs4all.nl/\~peace] \from[aa]

\Topic {Networking Warungs}

\startquotation
While the networking form of  social organization such as the warung
has  existed in  other times  and  spaces, the  warnet paradigm  has
provided  the basis for  its near-simultaneous  expansion throughout
the entire social structure.   [\dots ]  The nodes of cyberspace-warnet
and warung-like  settings joined to create a  powerful network that,
in the  case of Indonesia at  the end of the  twentieth century, was
more  dynamic than  the collapsing  networks of  the state-corporate
economy. \footnote{Merlyna  Lim (2003)  The Internet,  Social Network  and  Reform in
Indonesia}
\stopquotation

\placefigure[][]{}{\externalfigure[fotos/warnet_indonesia/servis_elektronik.jpg]}

Castells  refers to  this  out  come as  ``{\em the  pre-eminence of  social
morphology  over  social  action}''  a  pre-eminence that  is  the  main
characteristic of network society.


\Topic {WSIS on Indonesia}

The  year 2003 witnesses  a significant  ICT development  in Indonesia
especially  at  policy  and  regulatory  level.

\placefigure[][]{}{\externalfigure[fotos/warnet_indonesia/child_with_computer.jpg]}

The  World Summit  on Information  Society in  December 2003  drives a
significant impact  tuning the policy and regulatory  framework of the
country to  meet the objectives of  WSIS, such as,  {\bf connecting half of
the country population to the Internet by 2015}.


\Topic {Grassroot ICT communities}

Most of  implementations are community based,  self-sustained and with
minimal   outside   support   including  government   support.    Most
notable\footnote{source:  Onno W.  Purbo,  lecturer at  Institute of  Technology in
Bandung}:

\placefigure[][]{}{\externalfigure[fotos/warnet_indonesia/people_reaching_tech.jpg]}

- \useURL[aa][http://www.groups.or.id][][http://www.groups.or.id] \from[aa]: community funded free mailing-list service;
in 2004 it served 2500+ mailing list with 65,000+ subscribes.

- \useURL[aa][http://www.ilmukomputer.com][][http://www.ilmukomputer.com] \from[aa]:  a WSIS award winner  that provide free
ICT knowledge to the Indonesian society, funded by the society.

- \useURL[aa][http://www.voipmerdeka.net][][http://www.voipmerdeka.net] \from[aa]: the Rebelnet  that run free VoIP service
for and by Indonesian people: one of the {\bf largest free VoIP networks in
the World}

\Topic {Recent policy changes}

- 5  January 2005:  ministry act that  unlicenses the use  2.4GHz wifi
frequencies,  after many  police raids  and tiring  online  \& physical
debate with the government.

\placefigure[][]{}{\externalfigure[fotos/warnet_indonesia/win_piracy.jpg]}

- Mid  2005: Police  raid to Indonesian  cybercafes (warnets)  in many
cities  and  take  computers,   arrest  the  operators  /  owners  for
interrogation of software piracy.


\Topic {Future concerns}

\startquotation
The  concern  is  shifting  to  the  corporate  economy.   This  new
juggernaut   can   potentially   depoliticize   cyber-exchanges   by
transforming civil society into little more than a sum of individual
consumers having no identity other  than the biggest name brands and
latest  corporate commodities.  Thus,  the Internet  may potentially
become a sanitized, homogenous medium whose main function is to sell
consumerism to people and people to advertisers. \footnote{Merlyna  Lim (2003) "The  Internet, Social  Network and  Reform in
Indonesia"}
\stopquotation

\placefigure[][]{}{\externalfigure[fotos/warnet_indonesia/corporate_gringos.jpg]}

\startquotation
Through  the shift  from state  to corporate  hegemony  the Internet
becomes the  gate through which privileged  private interest invaded
the public sphere. \footnote{Juergen  Habermas  (1991) "The  Structural  Transformation of  the
Public Sphere: An  Inquiry into a Category of  Bourgeois Society", MIT
Press.}
\stopquotation


\Topic {Active networks}

Mapping  violence  and  injustice  has  a  prominent  role  among  the
activities of all these  humanitarian organisations.

- \useURL[bb][http://www.kontras.org/][][Kontras]\from[bb]\footnote{http://www.kontras.org/}: Victim's Place for Solidarity Building in Fighting for Justice

- \useURL[bb][http://www.ikohi.blogspot.com/][][IKOHI]\from[bb]\footnote{http://www.ikohi.blogspot.com/}: Commission for the Disappeared and Victims of Violence

- \useURL[bb][http://www.wirantaprawira.de/ypkp/][][YPKP]\from[bb]\footnote{http://www.wirantaprawira.de/ypkp/}: Indonesian Institute for the study of 1965/1966 Massacre

\placefigure[][]{}{\externalfigure[fotos/warnet_indonesia/ypkp.jpg]}

The raise of  {\em locative media} and modern mapping  technologies open the
possibility to complete such historical documentation, while deepening
the social importance of such technologies.


\Topic {Terima Kasih}

\placefigure[][]{}{\externalfigure[images/dyne-big.png]}

Jaromil's musings on \useURL[aa][http://jaromil.dyne.org][][http://jaromil.dyne.org] \from[aa]

Indonesian diary on \useURL[aa][http://jaromil.dyne.org/journal/indonesia.html][][http://jaromil.dyne.org/journal/indonesia.html] \from[aa]

Thanks to {\bf NLNET} for supporting this research.

Selamat jalan!



\stoptext
